\documentclass[12pt]{exam}
\usepackage[portuguese]{babel}
\usepackage[utf8]{inputenc}
\usepackage{graphicx}
\usepackage{pdfpages}
% Please add the following required packages to your document preamble:
\usepackage{booktabs}

\graphicspath{{figuras/}}

\usepackage[margin=1in]{geometry}
\usepackage{amsmath,amssymb}
\usepackage{multicol}
\usepackage{textcomp,lmodern,listings}
\lstset{language=SQL,basicstyle=\ttfamily, columns=fullflexible, upquote, morekeywords={data,wrapper,library,language}}

\newcommand{\disciplina}{Banco de Dados}
\newcommand{\class}{\disciplina}
\newcommand{\term}{Prof. Igor Avila Pereira}
\newcommand{\bimestre}{2}
\newcommand{\valor}{4}
\newcommand{\examnum}{Ativ. Avaliada 2 - \bimestreº Bim. - Valor: \valor}
%\newcommand{\examdate}{xx}
%\newcommand{\timelimit}{4 horas}

\pagestyle{head}
\firstpageheader{}{}{}
\runningheader{\class} - {Página \thepage\ de \numpages}
\runningheadrule

% https://math.mit.edu/~psh/exam/examdoc.pdf
% caixinha em vez de bolinha
\checkboxchar{$\Box$}
% reposta com caixinha preta em vez de checked
\checkedchar{$\blacksquare$}
% mostrar as respostas
\printanswers

% trocar a palavra point(s) por nada
\pointname{}
% https://groups.google.com/g/latex-br/c/DHdjfnfN90M


\begin{document}

\noindent
\begin{tabular*}{\textwidth}{l @{\extracolsep{\fill}} r @{\extracolsep{6pt}} l}
\textbf{\class} & \textbf{Nome:} & \makebox[2in]{\hrulefill}   \\
\textbf{\examnum} & \textbf{Matrícula:} & \makebox[2in]{\hrulefill}   \\
%\textbf{\examnum} &&\\
%\textbf{\examdate} &&\\
\textbf{\term} &&\\
%\textbf{Duração: \timelimit}
\end{tabular*}\\
\rule[2ex]{\textwidth}{2pt}
\noindent
%rule[2ex]{\textwidth}{2pt}

\begin{questions}


\question[1.0] Construa as instruções DDL necessárias para construir um B.D capaz de armazenar as informações cadastrais dos funcionários de uma empresa.  \label{q:dados_cadastrais_funcionario} 

\begin{figure}[!ht]
    \centering
    \includegraphics[scale=0.6]{figuras/dados_cadastrais_funcionario.png}
    % \caption{Caption}
    \label{fig:dados_cadastrais_funcionario}
\end{figure}

% \vspace{10px}

\textbf{Observações:}

\begin{itemize}
    \item cpf do funcionário é um campo único;
    \item endereço é o campo textual atômico;
    \item Leve em consideração que o funcionário pode ter vários cargos  (ocupações) e dependentes.
    \item No próprio B.D deve ser possível consultar e armazenar os nomes de todos os cargos possíveis que um funcionário pode ter dentro %do organograma 
    da empresa;
    \item um dependente é dependente de um único funcionário;
    % \item ao longo de sua carreira, um funcionário pode ter tido diversos cargos (ocupações); 
\end{itemize}


\question[1.0] Implemente em JAVA um sistema de cadastro de funcionários para empresa da questão \ref{q:dados_cadastrais_funcionario}. 

% \vspace{10px}

\textbf{Observações:}

\begin{itemize}
    \item Utilize JAVA/JDBC e PostgreSQL
    \item Utilize a mesma base de dados criada na questão \ref{q:dados_cadastrais_funcionario};
    \item A utilização do padrão DAO é opcional;
    \item Pode usar qualquer IDE/Editor;
\end{itemize}

% \newpage

     \textbf{\underline{IMPORTANTE:}} 
     
    \begin{itemize} 
     \item Nesta questão, utilize \textbf{\underline{SOMENTE}} os seguintes campos de funcionário:
    %  \begin{itemize} 
        %  \item  
        \textbf{cpf},
        %  \item 
        \textbf{nome},
        %  \item 
        \textbf{estado civil},
        %  \item 
        \textbf{endereço},
        %  \item 
        \textbf{nacionalidade},
        %  \item 
        \textbf{rg},
        %  \item 
        \textbf{telefone} e 
        %  
        \textbf{sexo}: 
    %  \end{itemize}
        \begin{itemize}
            \item Lembrando, mais uma vez, que \textbf{endereço} é o campo textual (atômico);
            \item o \textbf{cpf} pode ser usado como chave primária. Entretanto, se achar necessário, crie um id. 
            \item Desconsidere, nesta questão, os demais campos da ficha de cadastro dos funcionários (ex: data de nascimento e data de admissão);
            \item Desconsidere também o cadastro de Cargos, Dependentes e Ocupações;
        \end{itemize}
\end{itemize}


\question[1.0] Construa as instruções DDL necessárias para construir um B.D capaz de armazenar as informações contidas nas fichas médicas de todos os pacientes de uma clínica. %Leve em consideração que o CRM do médico serve para identificá-lo unicamente. 
\label{q:fica_medica}

\begin{figure}[!ht]
    \centering
    \includegraphics[scale=0.4]{figuras/ficha_medica}
    % \caption{Caption}
    \label{fig:ficha_medica}
\end{figure}

\textbf{Observações:}

\begin{itemize}
    \item O \textbf{número do paciente}, o \textbf{número da consulta} e o \textbf{crm do médico} são campos \textbf{únicos};
    \item uma consulta é realizado por um paciente e um médico;
    \item um exame é de um paciente e está ligado a uma consulta (por meio do número da consulta);
    \item No próprio B.D deve ser possível consultar e armazenar todos os convênios disponíveis;
    \item No próprio B.D deve ser possível consultar e armazenar todos os exames disponíveis; %para um paciente realizar; 
    \item ao longo de sua vida, um paciente pode ter feito diversas consultas com diversos médicos e ter realizado inúmeros exames; 
\end{itemize}

% \question (1.0) Escolha um dos BD's desenvolvidos pelas questões   \ref{q:dados_cadastrais_funcionario} e \ref{q:fica_medica} e crie:

\question[1.0] Crie para o B.D da clínica da questão  \ref{q:fica_medica}:

\begin{itemize}
    \item 1 \textit{superuser}: \textbf{fulano} com senha \textbf{fulano};
    \item 1 usuário que pode \textbf{\underline{somente} consultar}: \textbf{ciclano} com senha 
    \textbf{ciclano};
    % (fazer SELECT) nas tabelas do B.D;
\end{itemize}



% \textbf{Dicas:} CREATE SCHEMA, SET search\underline{\hspace{0.3cm}}path TO;
% % \begin{itemize}
% %     \item \textbf{Obs:} respeite nomes de tabelas/colunas/schemas, restrições de integridade e definição de schemas.
% % \end{itemize}

% \begin{figure}[!htp]
% \centering
% \includegraphics[width=1\linewidth]{figuras/prova1-2bim_2022.png}
% \end{figure}



% \question (0.5) Construa um STORE PROCEDURE de validação para a coluna \textit{cpf} da tabela \textbf{telespectador}.

% \vspace{2px}

%  \textbf{Observações:}

%     \begin{itemize}
%         \item Será preciso acrescentar o STORE PROCEDURE de validação à cláusula (\textit{check}) da coluna \textit{cpf} (tabela \textbf{telespectador});
%     \end{itemize}

% \textbf{Dicas:}

%     \begin{itemize}
%   \item A função desenvolvida nas aulas pode ser usada. Se não possuir a função em mãos, será preciso implementá-la novamente.
   
%   \item Lembrando que 00000000000, 11111111111, 22222222222, 33333333333, 44444444444, 55555555555, 66666666666, 77777777777, 88888888888 e 99999999999 também são cpf's \textbf{INVÁLIDOS};
%   \end{itemize}
  
% \question (1.0) Sabendo que os Funcionários realizam diversos turnos, construa um STORED PROCEDURE que calcule o salário de um determinado funcionário mediante o número de horas trabalhadas por ele durante o mês atual (mês corrente). 

% \vspace{2px}

% \textbf{Observações:} 

% \begin{itemize}
%     \item Este STORED PROCEDURE deve ter como \textbf{parâmetros de entrada}: 
%     \begin{itemize}
%         \item o \textbf{id do funcionário} e,
%         \item o \textbf{valor da hora trabalhada} de acordo com o seu setor:
%         \begin{itemize}
% \item Funcionários da \underline{\textbf{LIMPEZA}} ganham R\$ 10 por hora trabalhada,
% \item Funcionários do \underline{\textbf{ATENDIMENTO}} ganham R\$ 15 e,
% \item Funcionários da \underline{\textbf{OPERAÇÃO}} ganham R\$ 20.
% \end{itemize}
%     \end{itemize}
%     \item Só vale turnos que começam e terminam no mesmo mês, ou seja, data\underline{\hspace{0.2cm}}hora\underline{\hspace{0.2cm}}entrada e data\underline{\hspace{0.2cm}}hora\underline{\hspace{0.2cm}}saida pertencem ao mesmo mês (mês atual);
% \item Utilize os seguintes \textbf{casos de teste}: 
% \begin{itemize}
%     \item 1 funcionário de cada setor: LIMPEZA, ATENDIMENTO e OPERAÇÃO e,
%     \item 2 turnos diferentes de trabalho para cada funcionário;
% \end{itemize} 
% \item \textbf{\underline{IMPORTANTE:}} Caso facilite o cálculo, é \textbf{permitido considerar somente horas inteiras/completas de cada turno de trabalho}: 
% \begin{itemize}
%     \item \textbf{Ex:} 
%     \begin{enumerate}
%         \item Em um turno que durou 3 horas, 30 minutos e 45 segundos, é permitido considerar somente 3 horas.
%         \item Em um turno de 04:59, é permitido considerar somente 4 horas.
%     \end{enumerate}
% \end{itemize}
% \end{itemize}

% \textbf{Dicas:} EXTRACT, FOR LOOP, CAST e etc.

% \question (1.0) Construa uma TRIGGER que, a cada novo funcionário, crie um novo turno de trabalho para este funcionário recém cadastrado.

% \vspace{2px}

% \textbf{Observações:}

% \begin{itemize}
%     \item 
%     Este novo turno de trabalho deve começar no dia seguinte ao dia atual do cadastramento/inserção e deve começar às 08:00 (oito da manhã).
% \item A sala onde este funcionário irá trabalhar em seu primeiro turno de trabalho deve ser escolhida aleatoriamente entre as salas já existentes. 
% % \item \textbf{Obs:} Os tipos das colunas da tabela \textbf{funcionario\underline{\hspace{0.3cm}}sala} (que representa o turno de trabalho) não devem ser alterados.
% \end{itemize}

% \textbf{Dicas:} ORDER BY RANDOM(), LIMIT, CURRENT\underline{\hspace{0.3cm}}DATE + INTERVAL '1 day'.

% \question (0.5) Construa uma \textit{view} que retorne o \textbf{nome da sala}, o \textbf{título do filme}, o \textbf{nome do telespectador}, a \textbf{data} e \textbf{hora} da sessão, o \textbf{corredor} e a \textbf{poltrona} escolhidos de cada ingressos vendido. \textbf{Obs:} retornar a coluna \textbf{data} no formato dd/mm/aaaa.

% % Ex:

% % \begin{table}[]
% % \begin{tabular}{@{}|c|c|c|c|c|c|c|@{}}
% % \toprule
% % \textbf{sala\_nome} & filme\_titulo & \textbf{telespectador\_nome} & \textbf{data} & \textbf{hora} & \textbf{corredor} & \textbf{poltrona} \\ \midrule
% % 4k                  & TOP GUN       & igor                         & 2022-06-03    & 08:00         & A                 & 1                 \\ \midrule
% % 4k                  & TOP GUN       & betito                       & 2022-06-03    & 08:00         & B                 & 15                \\ \bottomrule
% % \end{tabular}
% % \end{table}

% \vspace{2px}

% \textbf{Dicas:} CREATE VIEW, INNER JOIN.

\end{questions}

% \includepdf[pages=-]{material/regra-cpf.pdf}

\end{document}